%!TEX TS-program = xelatex
%!TEX encoding = UTF-8 Unicode
% Awesome CV LaTeX Template for Cover Letter
%
% This template has been downloaded from:
% https://github.com/posquit0/Awesome-CV
%
% Authors:
% Claud D. Park <posquit0.bj@gmail.com>
% Lars Richter <mail@ayeks.de>
%
% Template license:
% CC BY-SA 4.0 (https://creativecommons.org/licenses/by-sa/4.0/)
%


%-------------------------------------------------------------------------------
% CONFIGURATIONS
%-------------------------------------------------------------------------------
% A4 paper size by default, use 'letterpaper' for US letter
\documentclass[11pt, a4paper]{awesome-cv}

% Configure page margins with geometry
% \geometry{left=1.4cm, top=.8cm, right=1.4cm, bottom=1.8cm, footskip=.5cm}

% Specify the location of the included fonts
\fontdir[fonts/]

% Color for highlights
% Awesome Colors: awesome-emerald, awesome-skyblue, awesome-red, awesome-pink, awesome-orange
%                 awesome-nephritis, awesome-concrete, awesome-darknight
\colorlet{awesome}{awesome-red}
% Uncomment if you would like to specify your own color
% \definecolor{awesome}{HTML}{CA63A8}

% Colors for text
% Uncomment if you would like to specify your own color
% \definecolor{darktext}{HTML}{414141}
% \definecolor{text}{HTML}{333333}
% \definecolor{graytext}{HTML}{5D5D5D}
% \definecolor{lighttext}{HTML}{999999}

% Set false if you don't want to highlight section with awesome color
\setbool{acvSectionColorHighlight}{true}

% If you would like to change the social information separator from a pipe (|) to something else
\renewcommand{\acvHeaderSocialSep}{\quad\textbar\quad}


%-------------------------------------------------------------------------------
%	PERSONAL INFORMATION
%	Comment any of the lines below if they are not required
%-------------------------------------------------------------------------------
\name{Larry}{Diehl}
\position{{Formal Methods Researcher \& Engineer}}
\address{2614 SW Water Ave -- Portland, OR -- 97201}

\mobile{407-718-7665} 
\email{larrytheliquid@gmail.com}
\homepage{www.larrytheliquid.com}
\github{larrytheliquid}
% \stackoverflow{SO-id}{SO-name}
% \twitter{@twit}
% \skype{skype-id}
% \reddit{reddit-id}
% \extrainfo{extra informations}

%% \quote{``Must be the change that you want to see in the world."}


%-------------------------------------------------------------------------------
%	LETTER INFORMATION
%	All of the below lines must be filled out
%-------------------------------------------------------------------------------
% The company being applied to
\recipient
  {Future Research and Teaching Interests}
  {Trustworthy Software Technologies - Tallinn University of Technology}
% The date on the letter, default is the date of compilation
\letterdate{\today}
% The title of the letter
%% \lettertitle{Job Application for Software Engineer/Researcher}
% How the letter is opened
\letteropening{}
% How the letter is closed
\letterclosing{Sincerely,}
% Any enclosures with the letter
%% \letterenclosure[Attached]{Résumé}

%-------------------------------------------------------------------------------
\begin{document}

% Print the header with above personal informations
\makecvheader

% Print the footer with 3 arguments(<left>, <center>, <right>)
% Leave any of these blank if they are not needed
\makecvfooter
  {\today}
  {Larry Diehl~~~·~~~Cover Letter}
  {}

% Print the title with above letter informations
\makelettertitle

%-------------------------------------------------------------------------------
%	LETTER CONTENT
%-------------------------------------------------------------------------------
\begin{cvletter}

\lettersection{Future Research Interests}

My future research interests involve continuing my efforts to make
dependently typed languages more realistic for real-world use, via
extensions to type theories, new programming techniques within
existing theories, and practical design of languages based on
underlying theories. Thus far I have done research on predicative
Church-style theories (e.g. Agda and Coq), as well as less popular
Curry-style impredicative theories (e.g. Cedille). This has given me a
better understanding of the advantages and trade-offs between such
theories. In the future, I am interested in exploring similar
trade-offs between extensional (e.g. Nuprl) and higher-dimensional
(e.g. HoTT) theories, and looking for ``sweet spots'' in language
design that maximize usefulness while minimizing complexity.

Having worked on datatype-generic programming using initial-algebra
encodings of datatypes has given me some experience with algebraic
programming. However, I am interested in more deeply learning category
theory, so that I can be better at structuring programs and theories,
and identifying existing structures to avoid reinventing one-off
instances of categorical concepts. Doing research in Tallinn would be
an exciting opportunity to get guidance from experts in these areas
like Tarmo Uustalu. Finally, I am also interested in looking into
applying my previous research to problem domains from industry (like
those that I worked on before my Ph.D.) to see how much the practical
gap has closed and what areas still need to be improved.

\lettersection{Teaching Interests}

Areas of Computer Science such as algorithms, networks, machine
learning, and databases have clear applications to industry, thus
students tend to have a natural motivation to learn such topics.
In contrast, the areas of programming languages and functional
programming have had much less adoption in industry. With my
background in working for Silicon Valley software companies, I feel
that I have a rare perspective of knowing the joy and pain involved in
programming with conventional (industrial) languages and functional
(academic) languages. I am excited by the idea of not only teaching
students in the areas of PL and FP, but also inspiring them with
concrete examples of how techniques from these areas elegantly address
issues from industry in ways that are mostly unknown.

\end{cvletter}


%-------------------------------------------------------------------------------
% Print the signature and enclosures with above letter informations
\makeletterclosing

\end{document}
