%!TEX TS-program = xelatex
%!TEX encoding = UTF-8 Unicode
% Awesome CV LaTeX Template for Cover Letter
%
% This template has been downloaded from:
% https://github.com/posquit0/Awesome-CV
%
% Authors:
% Claud D. Park <posquit0.bj@gmail.com>
% Lars Richter <mail@ayeks.de>
%
% Template license:
% CC BY-SA 4.0 (https://creativecommons.org/licenses/by-sa/4.0/)
%


%-------------------------------------------------------------------------------
% CONFIGURATIONS
%-------------------------------------------------------------------------------
% A4 paper size by default, use 'letterpaper' for US letter
\documentclass[11pt, a4paper]{awesome-cv}

% Configure page margins with geometry
% \geometry{left=1.4cm, top=.8cm, right=1.4cm, bottom=1.8cm, footskip=.5cm}

% Specify the location of the included fonts
\fontdir[fonts/]

% Color for highlights
% Awesome Colors: awesome-emerald, awesome-skyblue, awesome-red, awesome-pink, awesome-orange
%                 awesome-nephritis, awesome-concrete, awesome-darknight
\colorlet{awesome}{awesome-red}
% Uncomment if you would like to specify your own color
% \definecolor{awesome}{HTML}{CA63A8}

% Colors for text
% Uncomment if you would like to specify your own color
% \definecolor{darktext}{HTML}{414141}
% \definecolor{text}{HTML}{333333}
% \definecolor{graytext}{HTML}{5D5D5D}
% \definecolor{lighttext}{HTML}{999999}

% Set false if you don't want to highlight section with awesome color
\setbool{acvSectionColorHighlight}{true}

% If you would like to change the social information separator from a pipe (|) to something else
\renewcommand{\acvHeaderSocialSep}{\quad\textbar\quad}


%-------------------------------------------------------------------------------
%	PERSONAL INFORMATION
%	Comment any of the lines below if they are not required
%-------------------------------------------------------------------------------
\name{Larry}{Diehl}
\position{Software Engineer{\enskip\cdotp\enskip}Researcher}
\address{2614 SW Water Ave -- Portland, OR -- 97201}

\mobile{407-718-7665} 
\email{larrytheliquid@gmail.com}
\homepage{www.larrytheliquid.com}
\github{larrytheliquid}
% \stackoverflow{SO-id}{SO-name}
% \twitter{@twit}
% \skype{skype-id}
% \reddit{reddit-id}
% \extrainfo{extra informations}

%% \quote{``Must be the change that you want to see in the world."}


%-------------------------------------------------------------------------------
%	LETTER INFORMATION
%	All of the below lines must be filled out
%-------------------------------------------------------------------------------
% The company being applied to
\recipient
  {Job Application for Software Engineer/Researcher}
  {Galois Inc.\\421 SW 6th Avenue, Suite 300 \\Portland, Oregon 97204}
% The date on the letter, default is the date of compilation
\letterdate{\today}
% The title of the letter
%% \lettertitle{Job Application for Software Engineer/Researcher}
% How the letter is opened
\letteropening{}
% How the letter is closed
\letterclosing{Sincerely,}
% Any enclosures with the letter
%% \letterenclosure[Attached]{Résumé}

%-------------------------------------------------------------------------------
\begin{document}

% Print the header with above personal informations
\makecvheader

% Print the footer with 3 arguments(<left>, <center>, <right>)
% Leave any of these blank if they are not needed
\makecvfooter
  {\today}
  {Larry Diehl~~~·~~~Cover Letter}
  {}

% Print the title with above letter informations
\makelettertitle

%-------------------------------------------------------------------------------
%	LETTER CONTENT
%-------------------------------------------------------------------------------
\begin{cvletter}

\lettersection{Software Engineering Background}

My major professional software development experience has involved the
Ruby programming language.
The Ruby community emphasizes good software engineering practices,
like code quality, unit testing, integration testing, Test-Driven
Development, and Agile development.

The most prestigious Ruby company I worked for was Engine Yard,
where I worked on a cloud hosting platform built
on top of Amazon AWS. A monolithic application could not support such
a large product, so instead it was made of many isolated microservices
to tame the complexity. The main product involved the consumption of
these microservices, and careful orchestration was required to keep the
entire apparatus running smoothly. There I
realized the software engineering limits of a language like
Ruby for large software endeavours. While testing a single application
or microservice is easy, the combinatorial explosion of test cases
resulting from the interaction of microservices makes exhaustive
testing impossible, and reasonable coverage unreasonably difficult.

\lettersection{Formal Methods Background}

My experience working for Engine Yard made me look for other ways to
achieve better software engineering, ultimately settling on formal
methods. While unit tests can never exhaustively show correctness over
an infinite domain, a proof of a universally quantified proposition
can. This idea led me to write a proof of concept web framework
(Lemmachine) in the dependently typed Agda language. The idea was that
proofs of correctness for individual microservices could be reused by
correctness proofs of the consumer application because proofs are inherently
compositional.

I decided to a do a Ph.D. in Computer Science, hoping to
make such work more practical. I specialized in researching
dependently typed languages. While writing Lemmachine I encountered
code reuse issues with modern dependently typed languages, so my
research focused on generic programming to overcome such issues. Close
to graduating, I am now very comfortable with dependently typed
languages, both programming (and proving) with them and
implementing (using Haskell) custom versions of them to cater to
specific problems.

\lettersection{Why Galois?}

At the current point of my career I have experienced two extremes. The
first extreme was software engineering in Ruby for ordinary business
applications. The second extreme was academic formal methods
research using Haskell, Agda, and dependent types. I'm now ready to find the middle ground between the two,
finding the right compromises between engineering effort and desired
correctness criteria for specific projects. I believe Galois is a fantastic
fit for doing this.

\end{cvletter}


%-------------------------------------------------------------------------------
% Print the signature and enclosures with above letter informations
\makeletterclosing

\end{document}
