%!TEX TS-program = xelatex
%!TEX encoding = UTF-8 Unicode
% Awesome CV LaTeX Template for Cover Letter
%
% This template has been downloaded from:
% https://github.com/posquit0/Awesome-CV
%
% Authors:
% Claud D. Park <posquit0.bj@gmail.com>
% Lars Richter <mail@ayeks.de>
%
% Template license:
% CC BY-SA 4.0 (https://creativecommons.org/licenses/by-sa/4.0/)
%


%-------------------------------------------------------------------------------
% CONFIGURATIONS
%-------------------------------------------------------------------------------
% A4 paper size by default, use 'letterpaper' for US letter
\documentclass[11pt, a4paper]{awesome-cv}

% Configure page margins with geometry
% \geometry{left=1.4cm, top=.8cm, right=1.4cm, bottom=1.8cm, footskip=.5cm}

% Specify the location of the included fonts
\fontdir[fonts/]

% Color for highlights
% Awesome Colors: awesome-emerald, awesome-skyblue, awesome-red, awesome-pink, awesome-orange
%                 awesome-nephritis, awesome-concrete, awesome-darknight
\colorlet{awesome}{awesome-red}
% Uncomment if you would like to specify your own color
% \definecolor{awesome}{HTML}{CA63A8}

% Colors for text
% Uncomment if you would like to specify your own color
% \definecolor{darktext}{HTML}{414141}
% \definecolor{text}{HTML}{333333}
% \definecolor{graytext}{HTML}{5D5D5D}
% \definecolor{lighttext}{HTML}{999999}

% Set false if you don't want to highlight section with awesome color
\setbool{acvSectionColorHighlight}{true}

% If you would like to change the social information separator from a pipe (|) to something else
\renewcommand{\acvHeaderSocialSep}{\quad\textbar\quad}


%-------------------------------------------------------------------------------
%	PERSONAL INFORMATION
%	Comment any of the lines below if they are not required
%-------------------------------------------------------------------------------
\name{Larry}{Diehl}
\position{{Formal Methods Researcher \& Engineer}}
\address{2614 SW Water Ave -- Portland, OR -- 97201}

\mobile{407-718-7665} 
\email{larrytheliquid@gmail.com}
\homepage{www.larrytheliquid.com}
\github{larrytheliquid}
% \stackoverflow{SO-id}{SO-name}
% \twitter{@twit}
% \skype{skype-id}
% \reddit{reddit-id}
% \extrainfo{extra informations}

%% \quote{``Must be the change that you want to see in the world."}


%-------------------------------------------------------------------------------
%	LETTER INFORMATION
%	All of the below lines must be filled out
%-------------------------------------------------------------------------------
% The company being applied to
\recipient
  {Job Application for Post-Doc/Researcher Position}
  {Trustworthy Software Technologies - Tallinn University of Technology}
% The date on the letter, default is the date of compilation
\letterdate{\today}
% The title of the letter
%% \lettertitle{Job Application for Software Engineer/Researcher}
% How the letter is opened
\letteropening{}
% How the letter is closed
\letterclosing{Sincerely,}
% Any enclosures with the letter
%% \letterenclosure[Attached]{Résumé}

%-------------------------------------------------------------------------------
\begin{document}

% Print the header with above personal informations
\makecvheader

% Print the footer with 3 arguments(<left>, <center>, <right>)
% Leave any of these blank if they are not needed
\makecvfooter
  {\today}
  {Larry Diehl~~~·~~~Cover Letter}
  {}

% Print the title with above letter informations
\makelettertitle

%-------------------------------------------------------------------------------
%	LETTER CONTENT
%-------------------------------------------------------------------------------
\begin{cvletter}

\lettersection{Thesis Work}

I earned a Ph.D. in Computer Science from Portland State University,
with a research focus on datatype-generic programming (within a closed
predicative dependent type theory), culminating in my thesis, ``Fully
Generic Programming over Closed Universes of Inductive-Recursive
Types''. With the advent of formal proof assistants (or dependently typed
languages), constructive type theory can increasingly be used as a
programming language for writing certified software, rather than just
pen-and-paper mathematics. However, existing languages (e.g. Agda,
Coq, Idris, and Lean) are still far from being easy or practical
enough to be used by non-specialist programmers.

My Ph.D. work focused on extending dependently typed languages to
support writing generic functions (and generic correctness proofs
about them) once-and-for-all, that can be used with any datatype the
user could possibly define. In industry, such generic functions have
been popular in dynamic languages with runtime metaprogramming
facilities (e.g. Ruby), but without the ability to reason about their
correctness. One such example is generic marshalling and unmarshalling
functions (e.g. toJSON and fromJSON), and generic proofs about the
inverse properties about such functions (e.g. fromJSON . toJSON =
id). Such functions can be defined for any datatype the user may
declare, which introduces significant complexity in dependently typed
languages due to the presence of infinitary, indexed, and
inductive-recursive datatypes, which lie beyond the popular fragment
of algebraic datatypes that non-dependent functional languages
support.

Besides pragmatic motivations of making dependently typed languages
more usable, my interests also include the theory behind such
languages and other type theories. For example, my thesis (and papers
leading up to it) introduced a model of closed type theory whose
universe not only reflected codes for primitive types (pi, sigma, fix,
etc.), but also mutually closed codes for signature functors, whose
presence is crucial for the generic programming I achieved. My
theoretical interests also includes model theory and proof theory, and
their relationship to programming language semantics. For example, I
proved termination of hereditary substitution for canonical
System T via logical relations, whose inductive naturals cause the
standard proof-theoretic termination argument to fail.

\lettersection{Postdoc Work}

During my postdoc at the University of Iowa, my focus shifted from
predicative Church-style type theories to impredicative Curry-style
type theories (e.g. Cedille). This required me to reset many mental
assumptions about what types of datatype encodings and theorems are
possible in type theory. In particular, the work on such theories has
led to beautiful impredicative semantic accounts of datatypes as
initial algebras. In predicative type theories, initial algebras are
either phrased syntactically (via a datatype of descriptions for
signature functors, as in my thesis), or semantically by W-types, but
this requires moving from intentional type theory to extensional type
theory to preserve adequacy. The group at Iowa was already in the
process of discovering such semantic accounts before I came, but I
have been involved in extending them to larger classes of datatypes,
including inductive-inductive and inductive-recursive
ones. Additionally, I have worked on generic programming within this
semantic setting, focusing on zero-runtime-cost reuse of programs
and proofs between related datatypes, achieving software modularity
without performance penalties.

\lettersection{Software Engineering Work}

While the theory behind dependently typed languages is well understand
by researchers, the practical implementation of non-toy
implementations is a bit of a dark art. This is especially true when
it comes to features such as dependent pattern matching, and inferring
implicit arguments via dynamic higher-order unification. I am
fortunate enough to have worked on the implementation of dependently
typed languages during my Ph.D. (Spire and Ditto), and during my
postdoc (Cedille).

Before entering my Ph.D. program, I also acrued experience in industry
as a software engineer, working for startups in Florida and San
Francisco. At this time I primarily worked with dynamic languages
(Ruby) on problems in the medical space (Bear Den Designs), online
advertising space (IZEA), and cloud space (Engine Yard). The problem
areas I primarily worked on involved web applications, web services,
and distributed programming.

\end{cvletter}


%-------------------------------------------------------------------------------
% Print the signature and enclosures with above letter informations
\makeletterclosing

\end{document}
